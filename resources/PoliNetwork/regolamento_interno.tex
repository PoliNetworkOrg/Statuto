
\newpage

\section{Vecchi contenuti}


\subsection{Definizioni}

\begin{itemize}
 \item \textbf{Admin}: Admin è colui che...
\end{itemize}


\subsection{Il "Consiglio degli Admin"}

Il Consiglio degli Admin è l'organo democratico al vertice di PoliNetwork. È strutturato in due Camere, una aperta a tutti gli admin del network ed una riservata ai Consiglieri.
La divisione topica avviene seguendo le seguenti linee guida:

1) Obbligatoriamente: Discussione riguardo condotta degli admin e contestuali penalità. La decisione verrà riportata nella Camera Pubblica.
2) Facoltativamente: Discussioni riguardo materie sensibili o riservate (es. iniziative con alto rischio di essere copiate da terzi). Viene lasciata in questo caso facoltà di scelta al firmatario della proposta.


\subsection{Composizione del "Consiglio degli Admin"}

Del Consiglio fanno parte i rappresentanti degli Admin di ogni Corso di Studio con almeno tre Admin, scelti con voto palese di questi ultimi. In caso di parità verrà eletto il più giovane.

Il Consiglio è inoltre integrato da tutti gli amministratori, come uditori.


\subsection{Funzioni}

I componenti del Consiglio si occupano di gestire la comunicazione con il resto degli Admin del proprio corso di appartenenza, hanno inoltre la massima autonomia decisionale garantendo il rispetto del regolamento.



\subsection{Delegati}

Il Consiglio elegge inoltre con votazione segreta a maggioranza semplice un delegato per ciascuno dei seguenti ruoli, che sono delineati in seguito alla struttura:

\begin{itemize}


\item Gestione Progetti Innovativi e Supporto Tecnico (IT);
\item Comunicazione Esterna;
\item Eventi e Partnership;
\item Comunicazione Interna e Crescita.
\end{itemize}

I delegati opereranno nell'interesse del network e dei suoi princà¬pi, elencati alla sezione 08. L'elezione dei Delegati per ciascun ruolo avviene ogni anno, con possibilità di nuova elezione dei Delegati uscenti fino a un massimo di due mandati.



\subsection{Collaboratori dei delegati}
Ciascun delegato può poi avvalersi dell'aiuto di collaboratori esterni o interni, ed è sua responsabilità tenere coinvolto ed informato il Consiglio degli Admin.



\subsection{Decadenza, dimissioni e sfiducia}

I componenti del Consiglio degli Admin, in riferimento alla sezione 02, sono eletti all'interno dei Corsi di Studio con almeno tre Admin. Il mandato dura 12 mesi, a decorrere dalla data di inserimento nel Consiglio, con un massimo di tre mandati. L'elezione avviene secondo la sezione 02, e il vincitore verrà integrato nel Consiglio degli Admin non appena scadrà il mandato del consigliere precedente dello stesso corso di studi.
In caso di dimissioni di un membro del Consiglio o di un delegato, verrà svolta una nuova votazione. A seguito di comportamenti gravi e lesivi del network o del suo spirito, ogni membro del Consiglio o delegato può essere sfiduciato dal Consiglio, previo parere favorevole con voto segreto di almeno 2/3 + 1 dei membri dello stesso. Nel caso in cui il consigliere non sia anche delegato, può essere sfiduciato anche dai soli Admin del suo Corso di Studi, sempre con una maggioranza dei 2/3 + 1 ma con voto palese. Nel caso dei garanti il voto del Consiglio deve essere unanime e successivamente si procederà a nuova nomina del garante da parte del precedente allo sfiduciato.

La sezione viene applicata su proposta di un membro del Consiglio.
Ogni Responsabile Admin dovrà partecipare ad un incontro obbligatorio con Garanti e Delegati (non è necessaria la presenza di tutti) qualora almeno uno di loro ritenga che ci sia una mancanza di partecipazione in Consiglio. Durante la chiamata si deciderà se la mancanza del rappresentante è motivata o se procedere con l'applicazione della presente sezione.



\subsection{Admin del Network}

Ogni studente o Rappresentante in Consiglio di Corso di Studi interessato a diventare Admin può contattare il delegato alla Comunicazione Esterna, che riferirà al delegato alla Comunicazione Interna e Crescita. Quest'ultimo deciderà se procedere con la nomina, a seguito di un incontro tra il candidato Admin e uno o più dei delegati o garanti.

Requisito necessario per diventare ed essere Admin è essere provvisti di numero di matricola valido o, se in fase di transizione tra corsi di laurea, di risultare regolarmente iscritti al Politecnico di Milano. Eccezione è fatta per i gruppi extra del network, non inerenti a dei corsi di studio, dove è possibile nominare o confermare come Admin chiunque sia stato d'aiuto fino a quel momento in quei particolari gruppi.


Nel caso di corsi con due o meno Admin viene dato mandato al delegato alla Comunicazione Interna e Crescita di trovarne altri, per permettere al corso di essere inserito nel Consiglio.



\subsection{Principi e obiettivi del Network}

Gli obiettivi e i principi, parte integrante di questa struttura, si trovano in una pagina dedicata
\begin{itemize}


\item (link obiettivi)
\item (link principi)
\end{itemize}


\subsection{Dettagli Tecnici}

Il gruppo del Consiglio sarà su Telegram, piattaforma in grado di fornire strumenti come le votazioni, sia palesi che segrete.

Esiste un registro dei file e delle discussioni importanti, su GitLab, curato dal delegato IT in collaborazione con i membri del Consiglio.

Il codice dei progetti informatici del Network deve essere open source (sulle piattaforme GitHub o GitLab), per favorire la partecipazione e la trasparenza.



\subsection{Trasparenza}

Il nome dei membri del Consiglio e il loro tag Telegram sarà pubblicata sul sito web del Network.

I delegati sono tenuti a compilare una relazione a metà e fine mandato (quest'ultima volta in particolare a favorire un veloce e chiaro passaggio di consegne con il successore) che sono tenuti ad inviare al Consiglio e a tutti gli Admin. È possibile, a discrezione dei delegati, inviare una relazione ridotta a realtà esterne.



\subsection{Sviluppo dei corsi}

Per ogni Capo Admin, all'inizio, a metà mandato e prima della fine, viene organizzato un incontro con almeno uno dei due Garanti e il Delegato alla Comunicazione Interna e Crescita (o, in via eccezionale, un altro delegato). Nella prima riunione, il nuovo Responsabile presenta il suo piano per lo sviluppo del corso. Nella seconda, presenta il suo piano aggiornato e una breve relazione, e nell'ultima si parlerà di ciò che è stato raggiunto e delle sue opinioni opinioni riguardo al network.


